\part{Interface et tension interfaciale}

\chapter{Tension de surface}

Du blabla

\section{PLacer une image}

La figure \ref{fig:TourbillonsDeMarangoni} montre les tourbillons que l'on peut observer autour de l'écoulement de Marangoni.

\begin{figure}[!ht]
  \centering
  \includegraphics*[scale = .8]{./figures/chap1/Tourbillons_Marangoni.png}
  \caption{Photo des tourbillons générés en bordure de l'écoulement de Marangoni}
  \label{fig:TourbillonsDeMarangoni}
\end{figure}

Ici on place une citation \cite{le2021surface}

\clearpage
\section{Utilisation des cadres}


\begin{ombredef}
  \begin{definition}
    \textbf{Ceci est un cadre de couleur verte}:\bigskip

    Evironnement definition
  \end{definition}
\end{ombredef}


\begin{ombretheo}
  \begin{theo}
    \textbf{Ceci est un cadre de couleur rouge}:\bigskip

    Evironnement theoreme
  \end{theo}
\end{ombretheo}



\begin{ombreremarque}
  \begin{remarque}
    \textbf{Ceci est un cadre de couleur bleu}:\bigskip

    Evironnement remarque
  \end{remarque}
\end{ombreremarque}

